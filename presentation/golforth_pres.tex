\documentclass{beamer}

\usepackage[utf8]{inputenc}

%\usepackage[ngerman]{babel}
\usepackage{color}
\usepackage{colortbl}
\usepackage{scalefnt}
\usepackage{tikz}
\usepackage{amssymb}
\usetikzlibrary{arrows,shapes,snakes,positioning,fit,shadows,matrix,calc,patterns}
\usepackage{geometry}
\usepackage{pgfpages}
\usepackage{verbatim}
\usepackage{paralist}
\usepackage{listings}
\usepackage{multirow}
\usepackage{hhline}
\usepackage{hyphenat}

\usepackage{textpos}
\usepackage{environ}

\pgfdeclarelayer{background}
\pgfdeclarelayer{foreground}
\pgfsetlayers{background,main,foreground}


\definecolor{COL_DATA_CELL}{HTML}{2032B0}


%
% TIKZ STYLES AND DEFINES
%

\def\tinydist{0.4cm}    % defines the distance between global components
\def\smalldist{1.2cm}   % defines the distance between global components
\def\bigdist{1.8cm}     % defines the distance between products and global components

\tikzstyle{stylearrow} = [draw,>=latex']
\tikzstyle{labelnode} = []
\tikzstyle{stylenode} = [text width=2cm,fill=white,minimum height=1.5cm,rounded corners,text centered,draw]



\tikzstyle{stylenetwork} = [fill=white,ellipse,draw,inner sep=0.6cm,decoration=snake,decorate]
\tikzstyle{styleinterfaces} = [fill=white,draw,decorate]
      
\newcommand{\examplerequest}[1] { \vspace{3mm} \begin{tikzpicture} \node [request_example] {\texttt{#1} }; \end{tikzpicture}  }
\tikzstyle{request_example}=[draw, rounded corners,fill=white,inner sep=6pt,text width=10cm]




%
% header
%
\defbeamertemplate{headline}{pres header}{%
    \vspace{5mm}
    \begin{tikzpicture} 

    \node [draw=none,fill=white,text width=12.5cm,inner sep=0pt,minimum height=0.6cm] (headernode) { }; % the right one

    \path (headernode.west) node [anchor=west,xshift=10pt] { \Large{{\textcolor{black}{ \insertsection } }}};

%    \path (headernode.east) node [anchor=east] { \includegraphics[width=4cm]{gfx/logo_web_transp_fontblue.pdf} };

    \path [draw] (headernode.south west) -- (headernode.south east);


\end{tikzpicture} 

}

\defbeamertemplate{headline}{my empty header}{%
}




%
% basic tikz frame for each slide
%
\NewEnviron{tikzframe}{%
\begin{frame} 
\vspace{1cm} 
\begin{center} 
\begin{tikzpicture} 
\node [minimum width=10.5cm,minimum height=7.2cm,draw=none] (framenode) {};  
\BODY
\end{tikzpicture} 
\end{center} 
\end{frame}}

%
% subtext for tikzframes
\newcommand\tikzframesubtext[1]{
    \path (framenode.south) node [anchor=south] {#1}; 
}




\newcommand{\zwischenueberschrift}[1] { 
    \Huge{#1}
}

\newcommand{\ueberschrift}[1] { 
    \textbf{#1}
}





%
% DOCUMENT
%


\begin{document}




\setbeamertemplate{headline}[pres header]



%
% defs
%






\section{Golforth}

\begin{frame}
\begin{center}
\zwischenueberschrift{Forth Golfscript Interpreter}
\end{center}
\end{frame}


\begin{frame}
\begin{center}
\zwischenueberschrift{Golfscript}
\end{center}
\end{frame}


\begin{frame}
    
    \vspace{0.5cm}

    Code Golf 
    \begin{itemize}
    \item shortest possible source code that implements an algorithm  
    \item solving problems (holes) in as few keystrokes as possible
    \end{itemize}

    \vspace{0.5cm}
    \pause 

    Golfscript
    \begin{itemize}
        \item stack oriented, variables exist
        \item single symbols represent high level operations  
        \item strong typed
        \item heavy use of operator overloading and type coercion
    \end{itemize}
\end{frame}


\begin{frame}
    
    \vspace{0.5cm}

    Golfscript Types
    \begin{itemize}
        \item Integer: \texttt{1 2}
        \item Arrays:  \texttt{[1 2 3] [3]}
        \item Strings: \texttt{"one two three"}
        \item Blocks:  \texttt{\{1+\}}
    \end{itemize}

    \vspace{0.5cm}
    \pause 

    Golfscript Operator Example
    \begin{itemize}
        \item \texttt{12 3 * -> 36}
        \item \texttt{[50 51 52]' '* -> "50 51 52"}
    \end{itemize}


\end{frame}


\begin{frame}
\begin{center}
\zwischenueberschrift{Forth Implementation}
\end{center}
\end{frame}


\begin{frame}
    Typesystem
    \begin{itemize}
        \item typisierter ausdruck ist scalar am stack
        \item Anonyme functions (words) vs Storage
        \item werden anonyme funktionen wieder geloescht wenn sie nicht referenziert werden?
        \item vorteil speicher: es kommen nur benoetigte elemente auf den stack (zb nur den typ fetchen)
        \item garbage collector vom ertl?
    \end{itemize}


    \vspace{0.5cm}
\texttt{12 anon\_int s" 1 anon\_int golf\_+" anon\_block -> 2 elements on stack}

\end{frame}

\begin{frame}
    Arrays

    \begin{itemize}
        \item Not in syntax, operator
        \item code
    \end{itemize}
\end{frame}

\begin{frame}
    Blocks

    \begin{itemize}
        \item Stored as already translated strings
        \item code
    \end{itemize}
\end{frame}


\begin{frame}
    Preprocessor
    \begin{itemize}
        \item translates golfscript to forth based intermediate 
        \item based on regular expression of reference implementation 
        \item Responsible for:
        \begin{itemize}
            \item infer initial type from syntax
            \item symbol table for variables
        \end{itemize}
    \end{itemize}


    \vspace{0.5cm}

\begin{center}
    \small{    
\texttt{"2 \{1+\}:x"}  \\
$\downarrow$ \\
\texttt{2 anon\_int s" 1 anon\_int golf\_+" anon\_block create golf\_x , }
}
\end{center}

\end{frame}





\begin{frame}
    Type Coercion and Overloading

    \begin{itemize}
        \item einstellige operation fuer overloading
        \item zweistellige mit coerce
        \item hinweis das hier jede menge operationen in forth zu coden sind, 
              evtl nicht vollstaendig
    \end{itemize}
\end{frame}


\begin{frame}
    Conditionals and Loops

    \begin{itemize}
        \item functions which take code blocks as arguments
    \end{itemize}
\end{frame}


\begin{frame}
    Cutbacks
    \begin{itemize}
    \item Error Handling differs
    \item Probably not all operators implemented
    \end{itemize}
\end{frame}


\begin{frame}
    Usage of Idiomatic Forth
    \begin{itemize}
    \item stack paradigma mapped to typed language
    \item Macros?
    \end{itemize}
\end{frame}



\end{document}
